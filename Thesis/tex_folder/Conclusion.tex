\chapter{Conclusion}
In this research,we executed a comprehensive study on the time-series analysis of the popular topic models DTM and LDA. Our research focused on the fundamental time-series information of topic drifting and topic popularity.To compare DTM and LDA, we tried to extract this information from the topic distributions of both models. Multiple datasets along with multiple topic configurations were used for this experiment. Our findings are:

\begin{enumerate}
\item DTM takes 100 times longer to train the model as compared than LDA for large datasets.

\item Topic drifting is a unique property of DTM that is difficult to extract from the LDA model, but some datasets like Twitter do not have topic transition information, so applying DTM to such datasets is a waste of resources.

\item Time-series topic popularity can be extracted from both models, but it is precise from LDA because DTM has topic transition embedded in the topics.
\end{enumerate}

 Fragmentation of topics was also detected in this process from the datasets focused on one domain, e.g., "NeurIPS", which is another interesting aspect of this research and could be studied in the future. To summarize, time-series topic popularity --common information needed as time series information-- should be extracted from LDA because it is faster and provides concrete information as compared to DTM. However, if topic drifting is required, then DTM is the only option, although it sometimes may give inaccurate information.