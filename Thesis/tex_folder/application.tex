\chapter{Use Case in Social Media Analysis}
News extraction from Twitter data is a hot topic. But can we extract much more than just news? The purpose of this experiment is to find, either news is the only information which can be extracted from Twitter data or it contains much more insights about real life events. So, we used proposed  method (LDA topic popularity) for analysis of Twitter’s raw content. After pre-processing of tweets data, we apply hashtag pooling and extract topics using LDA without modifying its core machinery. In the second part, estimated number of tweets per day and correlated top hashtags for each topic are calculated using day-hashtag pooling. Finally, time series topic popularity graphs are constructed for topic analysis. Interesting results of bursty news detection, topic popularity, people’s way to perceiving an event, real-life event’s transition over time and before & after affects of a specific event were found.

\section{How Important is Twitter?}
Twitter is a very unique source of information where millions of users try to sum up an event, trend or their emotions into 140 characters. Diverse users of twitter freely express their thoughts which leads to many topics. Extracting trends from tweet’s data could be very handy to know and understand better about real-life events because of huge dataset available and people’s interest in it. The application area of twitter is vast including many useful domains such as real-time events detection \cite{sakaki2010earthquake}, sentiment predication analysis \cite{tumasjan2010predicting}, understanding public health opinions \cite{karami2018characterizing}, time series topic popularity variation \cite{fukuyama2018extracting} and it's comparison with traditional media \cite{zhao2011comparing}. Over 85\% of topics are headline news or persistent news in nature when tweets data is classified for trending topics  \cite{kwak2010twitter}. These topics aren't just only news but also contain reasons and effects of specific events. Also, people's interest is directly proportional to intensity of a specific event and its effects on people's life. As we know millions of tweets are tweeted everyday so it is impossible to extract topics manually. Twitter has hashtag information to follow the trending topics and frequency of tweets per hashtag can give us some information about popularity of a hashtag. But, hashtag is a user generated string and can lead to many topics or sometimes irrelevant information related to one specific topic. So, we used time-series topic popularity concept to find most of the useful information from twitter's raw data.

LDA is widely used for text classification of documents to topics but, it is not very efficient for short text documents, so hashtag pooling is used for making relatively big documents. Another problem is, LDA doesn't model variation of topics so we need a way to see the topics variation in time series manner to follow trends. Therefore, we introduce a way to transform LDA generated topics of twitter's data to time series trend analysis along with finding the correlations of these topics with hashtags.